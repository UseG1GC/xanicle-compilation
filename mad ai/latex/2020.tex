\documentclass[amssymb,twocolumn,pra,10pt,aps,nofootinbib]{revtex4-1}
\usepackage{mathptmx,amsmath, multirow}

\begin{document}
\title{The 81st William Lowell Putnam Mathematical Competition \\
    Saturday, February 20, 2021}
\maketitle

\begin{itemize}

\item[A1]
How many positive integers $N$ satisfy all of the following three conditions?
\begin{enumerate}
\item[(i)] $N$ is divisible by 2020.
\item[(ii)] $N$ has at most 2020 decimal digits.
\item[(iii)] The decimal digits of $N$ are a string of consecutive ones followed by a string of consecutive zeros.
\end{enumerate}

\item[A2]
Let $k$ be a nonnegative integer. Evaluate
\[
\sum_{j=0}^k 2^{k-j} \binom{k+j}{j}.
\]

\item[A3]
Let $a_0 = \pi/2$, and let $a_n = \sin(a_{n-1})$ for $n \geq 1$. Determine whether
\[
\sum_{n=1}^\infty a_n^2
\]
converges.

\item[A4]
Consider a horizontal strip of $N+2$ squares in which the first and the last square are black and the remaining $N$ squares are all white. Choose a white square uniformly at random, choose one of its two neighbors with equal probability,
and color this neighboring square black if it is not already black. Repeat this process until all the remaining white squares have only black neighbors. Let $w(N)$ be the expected number of white squares remaining. Find
\[
\lim_{N \to \infty} \frac{w(N)}{N}.
\]
 
\item[A5]
Let $a_n$ be the number of sets $S$ of positive integers for which
\[
\sum_{k \in S} F_k = n,
\]
where the Fibonacci sequence $(F_k)_{k \geq 1}$ satisfies $F_{k+2} = F_{k+1} + F_k$ and begins $F_1 = 1, F_2 = 1, F_3 = 2, F_4 = 3$. Find the largest integer $n$ such that $a_n = 2020$.

\item[A6] 
For a positive integer $N$, let $f_N$\footnote{Corrected from $F_N$ in the source.} be the function defined by 
\[
f_N(x) = \sum_{n=0}^N \frac{N+1/2-n}{(N+1)(2n+1)} \sin((2n+1)x).
\]
Determine the smallest constant $M$ such that $f_N(x) \leq M$ for all $N$ and all real $x$.

\item[B1]
For a positive integer $n$, define $d(n)$ to be the sum of the digits of $n$ when written in binary (for example, $d(13) = 1+1+0+1=3)$. Let
\[
S = \sum_{k=1}^{2020} (-1)^{d(k)} k^3.
\]
Determine $S$ modulo 2020.

\item[B2]
Let $k$ and $n$ be integers with $1 \leq k < n$. Alice and Bob play a game with $k$ pegs in a line of $n$ holes. At the beginning of the game, the pegs occupy the $k$ leftmost holes. A legal move consists of moving a single peg
to any vacant hole that is further to the right. The players alternate moves, with Alice playing first. The game ends when the pegs are in the $k$ rightmost holes, so whoever is next to play cannot move and therefore loses. For what values
of $n$ and $k$ does Alice have a winning strategy?

\item[B3]
Let $x_0 = 1$, and let $\delta$ be some constant satisfying $0 < \delta < 1$. Iteratively, for $n=0,1,2,\dots$, a point $x_{n+1}$ is chosen uniformly from the interval $[0, x_n]$. Let $Z$ be the smallest value of $n$ for which $x_n < \delta$.
Find the expected value of $Z$, as a function of $\delta$.

\item[B4]
Let $n$ be a positive integer, and let $V_n$ be the set of integer $(2n+1)$-tuples $\mathbf{v} = (s_0, s_1, \cdots, s_{2n-1}, s_{2n})$ for which $s_0 = s_{2n} = 0$ and $|s_j - s_{j-1}| = 1$ for $j=1,2,\cdots,2n$. Define
\[
q(\mathbf{v}) = 1 + \sum_{j=1}^{2n-1} 3^{s_j},
\]
and let $M(n)$ be the average of $\frac{1}{q(\mathbf{v})}$ over all $\mathbf{v} \in V_n$. Evaluate $M(2020)$.

\item[B5]
For $j \in \{1, 2, 3, 4\}$, let $z_j$ be a complex number with $|z_j| = 1$ and $z_j \neq 1$. Prove that
\[
3 - z_1 - z_2 - z_3 - z_4 + z_1 z_2 z_3 z_4 \neq 0.
\]

\item[B6]
Let $n$ be a positive integer. Prove that
\[
\sum_{k=1}^n (-1)^{\lfloor k(\sqrt{2}-1) \rfloor} \geq 0.
\]
(As usual, $\lfloor x \rfloor$ denotes the greatest integer less than or equal to $x$.)

\end{itemize}

\end{document}
