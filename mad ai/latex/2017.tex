\documentclass[amssymb,twocolumn,pra,10pt,aps]{revtex4-1}
\usepackage{mathptmx,amsmath, multirow}

\begin{document}
\title{The 78th William Lowell Putnam Mathematical Competition \\
    Saturday, December 2, 2017}
\maketitle

\begin{itemize}

\item[A1] 
Let $S$ be the smallest set of positive integers such that
\begin{enumerate}
\item[(a)]
$2$ is in $S$,
\item[(b)]
$n$ is in $S$ whenever $n^2$ is in $S$, and
\item[(c)]
$(n+5)^2$ is in $S$ whenever $n$ is in $S$.
\end{enumerate}
Which positive integers are not in $S$?

(The set $S$ is ``smallest'' in the sense that $S$ is contained in any other such set.)

\item[A2]
Let $Q_0(x) = 1$, $Q_1(x) = x$, and
\[
Q_n(x) = \frac{(Q_{n-1}(x))^2 - 1}{Q_{n-2}(x)}
\]
for all $n \geq 2$. Show that, whenever $n$ is a positive integer, $Q_n(x)$ is equal to a polynomial with integer coefficients.

\item[A3]
Let $a$ and $b$ be real numbers with $a<b$, and let $f$ and $g$ be continuous functions from $[a,b]$ to $(0, \infty)$
such that $\int_a^b f(x)\,dx = \int_a^b g(x)\,dx$ but $f \neq g$. For every positive integer $n$, define
\[
I_n = \int_a^b \frac{(f(x))^{n+1}}{(g(x))^n}\,dx.
\]
Show that $I_1, I_2, I_3, \dots$ is an increasing sequence with $\lim_{n \to \infty} I_n = \infty$.

\item[A4]
A class with $2N$ students took a quiz, on which the possible scores were $0,1,\dots,10$. Each of these scores
occurred at least once, and the average score was exactly $7.4$. Show that the class can be divided into two groups of $N$ students in such a way that the average score for each group was exactly $7.4$.

\item[A5]
Each of the integers from $1$ to $n$ is written on a separate card, and then the cards are combined into a deck and shuffled. Three players, $A$, $B$, and $C$, take turns in the order $A,B,C,A,\dots$ choosing one card at random from the deck. (Each card in the deck is equally likely to be chosen.) After a card is chosen, that card and all higher-numbered cards are removed from the deck, and the remaining cards are reshuffled before the next turn. Play continues until one of the three players wins the game by drawing the card numbered $1$.

Show that for each of the three players, there are arbitrarily large values of $n$ for which that player has the highest probability among the three players of winning the game. 

\item[A6]
The 30 edges of a regular icosahedron are distinguished by labeling them $1,2,\dots,30$. How many different ways 
are there to paint each edge red, white, or blue such that each of the 20 triangular faces of the icosahedron has two edges of the same color and a third edge of a different color? [Note: the top matter on each exam paper included the logo of the Mathematical Association of America, which is itself an icosahedron.]

\item[B1]
Let $L_1$ and $L_2$ be distinct lines in the plane. Prove that $L_1$ and $L_2$ intersect if and only if, for every
real number $\lambda\neq 0$ and every point $P$ not on $L_1$ or $L_2$, there exist points $A_1$ on $L_1$ and $A_2$
on $L_2$ such that $\overrightarrow{PA_2} = \lambda \overrightarrow{PA_1}$.

\item[B2]
Suppose that a positive integer $N$ can be expressed as the sum of $k$ consecutive positive integers
\[
N = a + (a+1) +(a+2) + \cdots + (a+k-1)
\]
for $k=2017$ but for no other values of $k>1$. Considering all positive integers $N$ with this property,
what is the smallest positive integer $a$ that occurs in any of these expressions?

\item[B3]
Suppose that $f(x) = \sum_{i=0}^\infty c_i x^i$ is a power series for which each coefficient $c_i$ is $0$ or $1$.
Show that if $f(2/3) = 3/2$, then $f(1/2)$ must be irrational.

\item[B4]
Evaluate the sum
\begin{gather*}
\sum_{k=0}^\infty \left( 3 \cdot \frac{\ln(4k+2)}{4k+2} - \frac{\ln(4k+3)}{4k+3} - \frac{\ln(4k+4)}{4k+4} - \frac{\ln(4k+5)}{4k+5} \right) \\
= 3 \cdot \frac{\ln 2}{2} - \frac{\ln 3}{3} - \frac{\ln 4}{4} - \frac{\ln 5}{5}
+ 3 \cdot \frac{\ln 6}{6} - \frac{\ln 7}{7} \\ - \frac{\ln 8}{8} - \frac{\ln 9}{9}
+ 3 \cdot \frac{\ln 10}{10} - \cdots .
\end{gather*}
(As usual, $\ln x$ denotes the natural logarithm of $x$.)

\item[B5]
A line in the plane of a triangle $T$ is called an \emph{equalizer} if it divides $T$ into two regions having equal area and equal perimeter. Find positive integers $a>b>c$, with $a$ as small as possible, such that there exists a triangle with side lengths $a, b, c$ that has exactly two distinct equalizers.

\item[B6]
Find the number of ordered $64$-tuples $(x_0,x_1,\dots,x_{63})$ such that $x_0,x_1,\dots,x_{63}$ are distinct elements of $\{1,2,\dots,2017\}$ and 
\[
x_0 + x_1 + 2x_2 + 3x_3 + \cdots + 63 x_{63}
\]
is divisible by 2017.
\end{itemize}

\end{document}
